We have fully developed the detector’s software and hardware components on our own. Therefore, precise calibration would require exposure to a controlled muon beam. Further, a controlled muon beam would permit us to test out features we are currently developing such as measuring the detection time and using a time-over-threshold system to find the energy of muons passing through the detector. Access to CERN’s facilities would allow us to validate our detector’s performance and refine its accuracy.

The confirmation of the viability at CERN of our safety-oriented detector design allows for open-sourcing, along with physics classroom outreach—projects with which we align. We believe that getting this detector ready for research will inspire more students to design their own particle physics experiments.
